\documentclass[a4paper,12pt]{article}
\usepackage[utf8]{inputenc}
\usepackage[margin=2cm]{geometry}
\usepackage{amsmath,amsthm,amsfonts,graphicx}
\usepackage{tikz}
\usepackage{xcolor}
\usepackage{hyperref}
\usepackage[capitalize]{cleveref}
%opening
%%%%%%%%%%%%%%%%%%%%%%%%%%%%%%%%%%%%%%%%%%%%%%%%%%%%%%%%%%%%%%%%%%%%%%%%%%%%%%%
\newcommand{\bs}[1]{\boldsymbol{#1}}
\newcommand{\Kref}{K_{\mathrm{ref}}}
\newcommand{\area}{\operatorname{area}}
\newcommand{\len}{\operatorname{len}}
\newcommand{\ud}{\mathrm{d}}
\newcommand{\uD}{\mathrm{D}}
\newcommand{\uN}{\mathrm{N}}
\newcommand{\gN}{g_{\uN}}
\newcommand{\GammaN}{\Gamma_{\uN}}
\newcommand{\GammaD}{\Gamma_{\uD}}
\newcommand{\free}{^{\mathrm{free}}}
\newcommand{\fix}{^{\mathrm{fix}}}
\newcommand{\rot}{^\llcorner}
%%%%%%%%%%%%%%%%%%%%%%%%%%%%%%%%%%%%%%%%%%%%%%%%%%%%%%%%%%%%%%%%%%%%%%%%%%%%%%%
\newtheorem{theorem}{Theorem}[section]
\newtheorem{lemma}[theorem]{Lemma}
%%%%%%%%%%%%%%%%%%%%%%%%%%%%%%%%%%%%%%%%%%%%%%%%%%%%%%%%%%%%%%%%%%%%%%%%%%%%%%%
\title{Notes}
\author{William McLean}
\date{\today}
%%%%%%%%%%%%%%%%%%%%%%%%%%%%%%%%%%%%%%%%%%%%%%%%%%%%%%%%%%%%%%%%%%%%%%%%%%%%%%%
\begin{document}
\maketitle
\tableofcontents
%%%%%%%%%%%%%%%%%%%%%%%%%%%%%%%%%%%%%%%%%%%%%%%%%%%%%%%%%%%%%%%%%%%%%%%%%%%%%%%
\section{Triangular elements}

Let $\bs{a}_1$, $\bs{a}_2$, $\bs{a}_3$ be the vertices of a triangle
in~$\mathbb{R}^2$, ordered counterclockwise.  We assume that the vertices are
not colinear and hence that the vectors $\bs{a}_1-\bs{a}_3$ and
$\bs{a}_2-\bs{a}_3$ are linearly independent.  The
\emph{barycentric coordinates}~$(\xi_1,\xi_2,\xi_3)$ of a
point~$\bs{x}\in\mathbb{R}^2$ with respect to this triangle are defined by the
relations
\[
\bs{x}=\xi_1\,\bs{a}_1+\xi_2\,\bs{a}_2+\xi_3\,\bs{a}_3
\quad\text{and}\quad
\xi_1+\xi_2+\xi_3=1.
\]
\cref{fig: barycentric} illustrates how the level sets of $\xi_i$ are
parallel to the side of the triangle opposite to the vertex~$\bs{a}_i$.

\begin{figure}
\caption{Level sets of the barycentric coordinates~$(\xi_1,\xi_2,\xi_3)$ with
respect to a triangle with vertices $\bs{a}_1$, $\bs{a}_2$, $\bs{a}_3$ and
centroid~$\bs{c}=\tfrac13\bs{a}_1+\tfrac13\bs{a}_2+\tfrac13\bs{a}_3$.}
\label{fig: barycentric}
\begin{center}
\includegraphics[scale=0.75]{images/barycentric-crop.pdf}
\end{center}
\end{figure}

To express $\xi_i$ in terms of~$\bs{x}$, we introduce vectors $\bs{b}_1$,
$\bs{b}_2$, $\bs{b}_3\in\mathbb{R}^2$ given by
\begin{equation}\label{eq: b vectors}
\begin{bmatrix}\bs{b}_1&\bs{b}_2 \end{bmatrix}
    =\begin{bmatrix}(\bs{a}_1-\bs{a}_3)&(\bs{a}_2-\bs{a}_3)\end{bmatrix}^{-\top}
\quad\text{and}\quad
\bs{b}_1+\bs{b}_2+\bs{b}_3=\bs{0},
\end{equation}
where we have used the notation
$\bs{A}^{-\top}=(\bs{A}^{-1})^\top=(\bs{A}^\top)^{-1}$ for the inverse
transpose of a non-singular matrix~$\bs{A}$.  Recall that the \emph{centroid}
of the triangle is the point
$\bs{c}=\tfrac13\bs{a}_1+\tfrac13\bs{a}_2+\tfrac13\bs{a}_3$, or in other words
the point with barycentric coordinates~$(\tfrac13,\tfrac13,\tfrac13)$.

\begin{lemma}\label{lem: xi_i}
The barycentric coordinates of~$\bs{x}$ are
\[
\xi_i=\tfrac13+\bs{b}_i\cdot(\bs{x}-\bs{c})\quad\text{for $i\in\{1,2,3\}$.}
\]
\end{lemma}
\begin{proof}
Since $\xi_3=1-\xi_1-\xi_2$,
\begin{equation}\label{eq: affine xi x}
\bs{x}=\xi_1\bs{a}_1+\xi_2\bs{a}_2+(1-\xi_1-\xi_2)\bs{a}_3
    =\bs{a}_3+\xi_1(\bs{a}_1-\bs{a}_3)+\xi_2(\bs{a}_2-\bs{a}_3)
\end{equation}
and so
\[
\begin{bmatrix}(\bs{a}_1-\bs{a}_3)&(\bs{a}_2-\bs{a}_3)\end{bmatrix}
\begin{bmatrix}\xi_1\\ \xi_2\end{bmatrix}=\bs{x}-\bs{a}_3.
\]
Multiplying on the left by $\begin{bmatrix}\bs{b}_1&\bs{b}_2\end{bmatrix}^\top$
gives
\[
\begin{bmatrix}\xi_1\\ \xi_2 \end{bmatrix}
    =\begin{bmatrix}\bs{b}_1^\top\\ \bs{b}_2^\top\end{bmatrix}(\bs{x}-\bs{a}_3)
    =\begin{bmatrix}\bs{b}_1\cdot(\bs{x}-\bs{a}_3)\\
                    \bs{b}_2\cdot(\bs{x}-\bs{a}_3)\end{bmatrix},
\]
showing that
\[
\xi_1=\bs{b}_1\cdot(\bs{x}-\bs{a}_3)\quad\text{and}\quad
\xi_2=\bs{b}_2\cdot(\bs{x}-\bs{a}_3).
\]
In particular, for the centroid~$\bs{c}$ we have
\[
\tfrac13=\bs{b}_1\cdot(\bs{c}-\bs{a}_3)=\bs{b}_2\cdot(\bs{c}-\bs{a}_3),
\]
and hence
\[
\xi_i=\bs{b}_i\cdot(\bs{x}-\bs{c}+\bs{c}-\bs{a}_3)
    =\tfrac13+\bs{b}_i\cdot(\bs{x}-\bs{c})\quad\text{for $i\in\{1,2\}$.}
\]
Finally,
\[
\xi_3=1-\xi_1-\xi_2=\tfrac13-(\bs{b}_1+\bs{b}_2)\cdot(\bs{x}-\bs{c})
    =\tfrac13+\bs{b}_3\cdot(\bs{x}-\bs{c}),
\]
so the formula holds also in the case~$i=3$.
\end{proof}

\cref{lem: xi_i} implies that if we view $\xi_i$ as a function of~$\bs{x}$, then
\begin{equation}\label{eq: grad xi_i}
\nabla\xi_i=\bs{b}_i\quad\text{for $i\in\{1,2,3\}$.}
\end{equation}
Hence, $\bs{b}_i$ must be normal to the level sets of~$\xi_i$, and in
particular to the side of the triangle opposite the vertex~$\bs{a}_i$.
Moreover, $\bs{b}_i$ points in the direction that $\xi_i$ increases, that is,
towards $\bs{a}_i$, as illustrated in \cref{fig: b vectors}.

\begin{figure}
\caption{The vectors $\bs{b}_1$, $\bs{b}_2$, $\bs{b}_3$ defined
in~\eqref{eq: b vectors}}\label{fig: b vectors}
\begin{center}
\includegraphics[scale=0.75]{images/b_vectors-crop.pdf}
\end{center}
\end{figure}

Let $K$ denote the triangle with vertices $\bs{a}_1$, $\bs{a}_2$, $\bs{a}_3$,
and define a reference element
\[
\Kref=\{\,(\xi_1,\xi_2):\text{$0\le\xi_1\le1$ and $0\le\xi_2\le1-\xi_1$}\,\}.
\]
We then have a bijective affine map from~$\Kref$ to~$K$, namely
$(\xi_1,\xi_2)\mapsto(x_1,x_2)$ given by~\eqref{eq: affine xi x}.  The Jacobian
determinant of this mapping is the constant
\[
\frac{\partial(x_1,x_2)}{\partial(\xi_1,\xi_2)}
=\det\begin{bmatrix}(\bs{a}_1-\bs{a}_3)&(\bs{a}_2-\bs{a_3})\end{bmatrix},
\]
and since $\area(\Kref)=\tfrac12$ it follows that
\[
2\area(K)=\det\begin{bmatrix}
    (\bs{a}_1-\bs{a}_3)&(\bs{a}_2-\bs{a_3})\end{bmatrix},
\]
so
\[
\int_Kf=2\area(K)\int_0^1\int_0^{1-\xi_1}f\bigl(\bs{x}(\xi_1,\xi_1)\bigr)
    \,\ud\xi_2\,\ud\xi_1.
\]
In particular, we have the following simple formula for integrating a monomial
in the barycentric coordinates over~$K$.

\begin{lemma}\label{lem: integral monomial}
For any non-negative integers $n_1$, $n_2$, $n_3$,
\[
\int_K\xi_1^{n_1}\xi_2^{n_2}\xi_3^{n_3}
    =2\area(K)\,\frac{n_1!\,n_2!\,n_3!}{(n_1+n_2+n_3+2)!}.
\]
\end{lemma}
\begin{proof}
Express the integral over~$K$ as
\begin{align*}
\int_K\xi_1^{n_1}\xi_2^{n_2}\xi_3^{n_3}&=2\area(K)\,n_1!\,n_2!\,n_3!
    \int_0^1\int_0^{1-\xi_1}\frac{\xi_1^{n_1}}{n_1!}\,\frac{\xi_2^{n_2}}{n_2!}\,
        \frac{\xi_3^{n_3}}{n_3!}\,\ud\xi_2\,\ud\xi_1\\
    &=2\area(K)\,n_1!\,n_2!\,n_3!
    \int_0^1\frac{\xi_1^{n_1}}{n_1!}\int_0^{1-\xi_1}
    \frac{\xi_2^{n_2}}{n_2!}\,
    \frac{(1-\xi_1-\xi_2)^{n_3}}{n_3!}\,\ud\xi_2\,\ud\xi_1.
\end{align*}
Repeated integration by parts shows that
\[
\int_0^a\frac{\xi^m}{m!}\,\frac{(1-\xi)^n}{n!}\,\ud\xi
    =\frac{a^{m+n}}{(m+n+1)!}
\]
so
\[
\int_0^{1-\xi_1}\frac{\xi_2^{n_2}}{n_2!}\,
    \frac{(1-\xi_1-\xi_2)^{n_3}}{n_3!}\,\ud\xi_2
    =\frac{(1-\xi_1)^{n_2+n_3}}{(n_2+n_3+1)!}
\]
and in turn
\begin{align*}
\int_0^1\frac{\xi_1^{n_1}}{n_1!}\int_0^{1-\xi_1}
    \frac{\xi_2^{n_2}}{n_2!}\,
    \frac{(1-\xi_1-\xi_2)^{n_3}}{n_3!}\,\ud\xi_2\,\ud\xi_1
    &=\int_0^1\frac{\xi_1^{n_1}}{n_1!}\,
    \frac{(1-\xi_1)^{n_2+n_3}}{(n_2+n_3+1)!}\,\ud\xi_1\\
    &=\frac{1}{(n_1+n_2+n_3+2)!},
\end{align*}
implying the desired formula.
\end{proof}
%%%%%%%%%%%%%%%%%%%%%%%%%%%%%%%%%%%%%%%%%%%%%%%%%%%%%%%%%%%%%%%%%%%%%%%%%%%%%%%
\section{Conforming linear elements}
Let $\psi_1$, $\psi_2$, $\psi_3$ be the linear shape functions for the
triangule~$K$, that is, polynomials of degree at most~$1$ in $x_1$~and $x_2$
that satisfy
\[
\psi_i(\bs{a}_j)=\delta_{ij}\quad\text{for $i$, $j\in\{1,2,3\}$.}
\]
By considering the level sets of the barycentric coordinates as
in~\cref{fig: barycentric}, we see that
\[
\psi_i(\bs{x})=\xi_i=\tfrac13+\bs{b}_i\cdot(\bs{x}-\bs{c})
    \quad\text{for $i\in\{1,2,3\}$.}
\]

Consider a Poisson equation,
\[
-\nabla\cdot(a\nabla u)=f,
\]
with a possibly variable coefficient~$a=a(\bs{x})$.  In a finite element
method using conforming P1 (piecewise-linear) elements, the entries~$A_{ij}$
of the $3\times3$ element stiffness matrix are
\[
A_{ij}=\int_K a\nabla\psi_j\cdot\nabla\psi_i=\bs{b}_j\cdot\bs{b}_i\int_K a.
\]
If $a$ is constant, then $\int_Ka=a\times\area(K)$.  For a variable
coefficient, we use the quadrature rule
\[
\int_Ka\approx\frac{\area(K)}{3}\sum_{i=1}^3 a(\bs{x}_i)
\]
where the quadrature points are
\begin{equation}\label{eq: quad points}
\begin{aligned}
\bs{x}_1&=(1-2s)\bs{a}_1+s\bs{a}_2+s\bs{a}_3,\\
\bs{x}_2&=s\bs{a}_1+(1-2s)\bs{a}_2+s\bs{a}_3,\\
\bs{x}_3&=s\bs{a}_1+s\bs{a}_2+(1-2s)\bs{a}_3,
\end{aligned}
\end{equation}
We see using \cref{lem: integral monomial} that, with any choice of~$s$,
\[
\int_K\xi_j=\frac{\area(K)}{3}=\frac{\area(K)}{3}\sum_{i=1}^3\xi_j(\bs{x}_i)
\quad\text{for $j\in\{1,2,3\}$,}
\]
and therefore the quadrature rule is exact for polynomials in~$\bs{x}=(x,y)$ of
degree~$1$ or less.  Also,
\[
\int_K\xi_j^2=\frac{\area(K)}{6}\quad\text{and}\quad
\frac{\area(K)}{3}\sum_{i=1}^3\xi_j(\bs{x}_i)^2=\frac{\area(K)}{6}(12s^2-8s+2),
\]
which are equal iff $12s^2-8s+2=1$, or equivalently
\begin{equation}\label{eq: s condition}
(6s-1)(2s-1)=0.
\end{equation}
Moreover, if $(j,k)\in\{(1,2),(2,3),(3,1)\}$ then
\[
\int_K\xi_j\xi_k=\frac{\area(K)}{12}\quad\text{and}\quad
\frac{\area(K)}{12}\sum_{i=1}^3\xi_j\xi_k=\frac{\area(K)}{3}(-12s^2+8s),
\]
which are equal iff $-12s^2+8s=1$, leading again to~\eqref{eq: s condition}.
Thus, the quadrature rule is exact for polynomials of degree~$2$ iff $s=1/6$~or
$s=1/2$.  These choices are illustrated in~\cref{fig: quad points}.

\begin{figure}
\caption{The quadrature points \eqref{eq: quad points} if $s=1/6$ (left) or
$s=1/2$ (right).}\label{fig: quad points}
\begin{center}
\includegraphics[scale=1.0]{images/quadrature_points-crop.pdf}
\end{center}
\end{figure}

The components of the element load vector are
\[
f_i=\int_Kf\psi_i\quad\text{for $i\in\{1,2,3\}$,}
\]
and if $f$ is constant then
\[
f_i=f\int_K\psi_i=f\int_K\xi_i=\tfrac13f\quad\text{for $i\in\{1,2,3\}$.}
\]
For a variable~$f$, we use the piecewise-linear approximation
\[
f(\bs{x})\approx\sum_{j=1}^3f(\bs{a}_j)\psi_i(\bs{x})
\]
and obtain
\[
\int_Kf\psi_i\approx\sum_{j=1}^3M_{ij}f(\bs{a}_j),
\]
where the entries of the mass matrix are
\begin{equation}\label{eq: Mij const}
M_{ij}=\int_K\psi_j\psi_i=\frac{\area(K)}{12}\times\begin{cases}
    2&\text{if $i=j$,}\\
    1&\text{if $i\ne j$.}
\end{cases}
\end{equation}
In the case of a mass matrix with a variable coefficient, we use
\begin{equation}\label{eq: Mij variable}
\int_Kc\psi_j\psi_i\approx\sum_{\ell=1}^3m_{ij\ell} c(\bs{a}_\ell)
    \quad\text{where}\quad m_{ij\ell}=\int_K\psi_\ell\psi_j\psi_i,
\end{equation}
and find that
\[
m_{ij\ell}=\frac{\area(K)}{60}\times\begin{cases}
    6&\text{if $\ell=i=j$,}\\
    2&\text{if $i=j\ne\ell$ or $i\ne j=\ell$ or $j\ne i=\ell$,}\\
    1&\text{if $i\ne j$ and $j\ne\ell$ and $i\ne\ell$.}
\end{cases}
\]

Now consider a boundary edge $E=[\bs{a}_1,\bs{a}_2]$ where a Neumann boundary
condition is imposed.  We introduce the barycentric coordinates~$(\xi_1,\xi_2)$
of a point~$\bs{x}\in\mathbb{R}^2$ with respect to~$E$ by the relations
\[
\bs{x}=\xi_1\bs{a}_1+\xi_2\bs{a}_2\quad\text{and}\quad\xi_1+\xi_2=1.
\]
The linear shape functions are then
\[
\psi^E_i(\bs{x})=\xi_i\quad\text{for $i\in\{1,2\}$,}
\]
and the entries of the boundary element load vector are
\[
g_{\uN i}=\int_Eg_{\uN}\psi^E_i\quad\text{for $i\in\{1,2\}$.}
\]
Arguing as in \cref{lem: integral monomial} we have
\[
\int_E\xi_1^{n_1}\xi_2^{n_2}=\len(E)\,\frac{n_1!n_2!}{(n_1+n_2+1)!},
\]
where $\len(E)=|\bs{a}_2-\bs{a}_1|$ denotes the length of~$E$,
so the approximation
\[
g_{\uN}(\bs{x})\approx g_{\uN}(\bs{a}_1)\psi^E_1(\bs{x})
+g_{\uN}(\bs{a}_2)\psi^E_2(\bs{x})\quad\text{for $\bs{x}\in E$}
\]
gives
\begin{equation}\label{eq: boundary element vector}
g_{\uN i}\approx\sum_{j=1}^2M^E_{ij}g_{\uN}(\bs{a}_j)
\quad\text{where}\quad M^E_{ij}=\frac{\len(E)}{6}\times\begin{cases}
    2&\text{if $i=j$,}\\
    1&\text{if $i\ne j$.}
\end{cases}
\end{equation}
For a boundary mass matrix with a variable coefficient,
\[
M^E_{ij}=\int_E c\psi^E_j\psi^E_i
\approx\sum_{\ell=1}^2m^E_{ij\ell}c(\bs{a}_\ell),
\]
where
\[
m^E_{ij\ell}=\int_E\psi_\ell\psi_i\psi_j=\frac{\len(E)}{24}\times\begin{cases}
    6&\text{if $i=j=\ell$,}\\
    2&\text{if $i=j\ne\ell$ or $i\ne j=\ell$ or $j\ne i=\ell$,}\\
    1&\text{if $i\ne j$ and $j\ne\ell$ and $i\ne\ell$.}
\end{cases}
\]


%%%%%%%%%%%%%%%%%%%%%%%%%%%%%%%%%%%%%%%%%%%%%%%%%%%%%%%%%%%%%%%%%%%%%%%%%%%%%%%
\section{Nonconforming linear elements}\label{sec: nonconforming}

\begin{figure}
\caption{Numbering of the vertices and midpoints.}\label{fig: vertex midpt}
\begin{center}
\begin{tikzpicture}[scale=0.75]
\draw[-] (0,0) -- (6,4) -- (-2,8) -- (0,0);
\draw[fill] (0,0) circle (0.05);
\node[below] at (0,0) {$\bs{a}_1$};
\draw[fill] (6,4) circle (0.05);
\node[right] at (6,4) {$\bs{a}_2$};
\draw[fill] (-2,8) circle (0.05);
\node[above] at (-2,8) {$\bs{a}_3$};
\draw[fill] (3,2) circle (0.05);
\node[below right] at (3,2) {$\bs{a}_4$};
\draw[fill] (2,6) circle (0.05);
\node[above right] at (2,6) {$\bs{a}_5$};
\draw[fill] (-1,4) circle (0.05);
\node[above right] at (-1,4) {$\bs{a}_6$};
\end{tikzpicture}
\end{center}
\end{figure}

Consider a non-conforming P1 finite element space for which we enforce
continuity only at the midpoints of the edges.  For the triangle~$K$ with
vertices $\bs{a}_1$, $\bs{a}_2$, $\bs{a}_3$ we denote the midpoints of the edges
by
\[
\bs{a}_4=\tfrac12\bs{a}_1+\tfrac12\bs{a}_2,\quad
\bs{a}_5=\tfrac12\bs{a}_2+\tfrac12\bs{a}_3,\quad
\bs{a}_6=\tfrac12\bs{a}_3+\tfrac12\bs{a}_1,
\]
consistent with the ordering used by Gmsh; see \cref{fig: vertex midpt}.  The
local degrees of freedom are the values of the finite element solution at the
edge midpoints
\[
\mathsf{n}_1=\bs{a}_4,\quad\mathsf{n}_2=\bs{a}_5,\quad\mathsf{n}_3=\bs{a}_6,
\]
so we require linear shape functions $\psi_1$, $\psi_2$, $\psi_3$ satisfying
\[
\psi_i(\mathsf{n}_j)=\delta_{ij}\quad\text{for $i$, $j\in\{1,2,3\}$.}
\]
The barycentric coordinates of the edge midpoints are
\[
\mathsf{n}_1\leftrightarrow(\tfrac12,\tfrac12,0),\quad
\mathsf{n}_2\leftrightarrow(0,\tfrac12,\tfrac12),\quad
\mathsf{n}_3\leftrightarrow(\tfrac12,0,\tfrac12),
\]
so, assuming that $\bs{x}$ has barycentric coordinates $(\xi_1,\xi_2,\xi_3)$,
\begin{equation}\label{eq: psi_i nonconforming}
\begin{aligned}
\psi_1(\bs{x})&=\xi_1+\xi_2-\xi_3=1-2\xi_3,\\
\psi_2(\bs{x})&=\xi_2+\xi_3-\xi_1=1-2\xi_1,\\
\psi_3(\bs{x})&=\xi_3+\xi_1-\xi_2=1-2\xi_2.
\end{aligned}
\end{equation}
Therefore, remembering that $\nabla\xi_i=\bs{b}_i$~and
$\bs{b}_1+\bs{b}_2+\bs{b_3}=\bs{0}$,
\begin{align*}
\nabla\psi_1(\bs{x})&=\bs{b}_1+\bs{b}_2-\bs{b}_3=-2\bs{b}_3,\\
\nabla\psi_2(\bs{x})&=\bs{b}_2+\bs{b}_3-\bs{b}_1=-2\bs{b}_1,\\
\nabla\psi_3(\bs{x})&=\bs{b}_3+\bs{b}_1-\bs{b}_2=-2\bs{b}_2.
\end{align*}
If we define $i_+$ and $i_-$ for $i\in\{1,2,3\}$ according to the table
\begin{center}
\renewcommand{\arraystretch}{1.2}
\begin{tabular}{c|c|c}
$i$&$i_+$&$i_-$\\
\hline
1&2&3\\
2&3&1\\
3&1&2
\end{tabular}
\end{center}
then
\begin{equation}\label{eq: grad psi_i}
\psi_i(\bs{x})=\xi_i+\xi_{i_+}-\xi_{i_-}=1-2\xi_{i_-}
\quad\text{and}\quad
\nabla\psi_i(\bs{x})=-2\bs{b}_{i_-}.
\end{equation}
Thus, the entries of the element stiffness matrix
\[
A_{ij}=\int_Ka\nabla\psi_j\cdot\nabla\psi_i
    =4(\bs{b}_{j_-}\cdot\bs{b}_{i_-})\int_Ka
\quad\text{for $i$, $j\in\{1,2,3\}$.}
\]

Suppose that the values of the finite element solution~$u_h$ at the midpoints
of the edges of~$K$ are
\[
U_i=u_h(\mathsf{n}_i)\quad\text{for $i\in\{1,2,3\}$,}
\]
so that
\[
u_h(\bs{x})=\sum_{i=1}^3U_i\psi_i(\bs{x})\quad\text{for $\bs{x}\in K$.}
\]
To determine the values of~$u_h$ at the vertices of~$K$, we observe that
\[
\psi_i(\bs{a}_j)=\xi_i(\bs{a}_j)+\xi_{i_+}(\bs{a}_j)-\xi_{i_-}(\bs{a}_j)
    =\delta_{ij}+\delta_{i_+j}-\delta_{i_-j}
\]
so
\[
u_h(\bs{a}_j)=\sum_{i=1}^3U_i\psi_i(\bs{a}_j)
    =U_j+U_{j_-}-U_{j_+}\quad\text{for $j\in\{1,2,3\}$.}
\]

Next, we consider the entries of the element mass matrix
\[
M_{ij}=\int_Kc\psi_j\psi_i.
\]
If the coefficient~$c$ is constant, then
\begin{equation}\label{eq: nonconforming mass matrix}
[M_{ij}]=\frac{c}{3}\,\area(K)
    \begin{bmatrix}1&0&0\\ 0&1&0\\ 0&0&1\end{bmatrix}.
\end{equation}
In fact, the diagonal entries are
\[
M_{ii}=c\int_K(1-2\xi_{i_-})^2
    =c\int_K(1-4\xi_{i_-}+4\xi_{i_-}^2)
\]
and, using \cref{lem: integral monomial},
\[
M_{ii}=c\area(K)\biggl(1-4\times2\times\frac{1}{3!}
    +4\times2\times\frac{2}{4!}\biggr)
    =\frac{c}{3}\,\area(K),
\]
whereas similar calculations show that $M_{12}=M_{23}=M_{31}=0$.  Alternatively,
\eqref{eq: nonconforming mass matrix} follows from the fact that the quadrature
rule
\begin{equation}\label{eq: edge midpt rule}
\int_K f\approx\frac{\area(K)}{3}\sum_{i=1}^3f(\mathsf{n}_i)
\end{equation}
is exact whenever $f$ is a quadratic polynomial.  If $c$ is variable, then
we use its linear interpolant with respect to the $\mathsf{n}_\ell$ to obtain
the approximation
\[
M_{ij}\approx\int_K\sum_{\ell=1}^3c(\mathsf{n}_\ell)\psi_\ell\psi_j\psi_i
    =\sum_{\ell=1}^3 m_{ij\ell}c(\mathsf{n}_\ell)
\quad\text{where}\quad m_{ij\ell}=\int_K\psi_\ell\psi_j\psi_i.
\]
We find that
\[
m_{ij\ell}=\frac{\area(K)}{15}\times\begin{cases}
    3&\text{if $i=j=k$,}\\
    1&\text{if $i=j\ne\ell$ or $i\ne j=\ell$ or $j\ne i=\ell$,}\\
    -2&\text{if $i\ne j$ and $j\ne\ell$ and$\ell\ne i$,}
\end{cases}
\]
so, putting $c_l=c(\mathsf{n}_\ell)$ for $\ell\in\{1,2,3\}$,
\begin{align*}
M_{11}\approx m_{111}c_1+m_{112}c_2+m_{113}c_3=\frac{\area(K)}{15}
(3c_1+c_2+c_3),\\
M_{22}\approx m_{221}c_1+m_{222}c_2+m_{223}c_3=\frac{\area(K)}{15}
(c_1+3c_2+c_3),\\
M_{33}\approx m_{331}c_1+m_{332}c_2+m_{333}c_3=\frac{\area(K)}{15}
(c_1+c_2+3_3),\\
\end{align*}
with
\begin{align*}
M_{12}&=M_{21}=m_{121}c_1+m_{122}c_2+m_{123}c_3=\frac{\area(K)}{15}
(c_1+c_2-2c_3),\\
M_{13}&=M_{31}=m_{131}c_1+m_{132}c_2+m_{133}c_3=\frac{\area(K)}{15}
(c_1-2c_2+c_3),\\
M_{23}&=M_{32}=m_{231}c_1+m_{232}c_2+m_{233}c_3=\frac{\area(K)}{15}
(-2c_1+c_2+c_3).
\end{align*}

Applying the rule~\eqref{eq: edge midpt rule} to $f\psi_i$ instead of just $f$
yields an approximation for the components of the element load vector:
\[
f_i=\int_Kf\psi_i\approx\frac{\area(K)}{3}\,f(\mathsf{n}_i)
\quad\text{for $i\in\{1,2,3\}$.}
\]
We obtain the same approximation for~$f_i$ by instead using linear
interpolation,
\[
f(\bs{x})\approx\sum_{j=1}^3f(\mathsf{n}_j)\psi_j(\bs{x})
    \quad\text{for $\bs{x}\in K$,}
\]
and then applying \eqref{eq: nonconforming mass matrix}:
\[
\int_Kf\psi_i\approx\sum_{j=1}^3f(\mathsf{n}_j)\int_K\psi_j\psi_i
    =\frac{\area(K)}{3}\,f(\mathsf{n}_i).
\]

\begin{figure}
\caption{Node labels for discussing a Neumann boundary condition. The boundary
edge~$E$ and the opposite vertex~$\bs{a}_k$ are shown in blue.}
\label{fig: Neumann bc}
\begin{center}
\begin{tikzpicture}[scale=0.75]
\draw[-] (-2,8) -- (0,0) -- (6,4);
\draw[-,thick,blue] (6,4) -- (-2,8);
\draw[fill,blue] (0,0) circle (0.075);
\node[below,blue] at (0,0) {$\bs{a}_k$};
\draw[fill] (6,4) circle (0.075);
\node[right] at (6,4) {$\bs{a}_{k_+}$};
\draw[fill] (-2,8) circle (0.075);
\node[above] at (-2,8) {$\bs{a}_{k_-}$};
\draw[fill] (3,2) circle (0.075);
\node[below right] at (3,2) {$\mathsf{n}_k$};
\draw[fill] (2,6) circle (0.075);
\node[above right] at (2,6) {$\mathsf{n}_{k_+}$};
\draw[fill] (-1,4) circle (0.075);
\node[above right] at (-1,4) {$\mathsf{n}_{k_-}$};
\end{tikzpicture}
\end{center}
\end{figure}

In the case of a Neumann boundary condition
\[
a\,\frac{\partial u}{\partial n}=\gN(\bs{x})\quad\text{for $\bs{x}\in\GammaN$,}
\]
we need to compute the entry of the boundary element load vector,
\[
g_i=\int_E\gN\psi_i\quad\text{for $1\le i\le 3$,}
\]
where the $\psi_i$ are the shape functions for the (unique) triangle~$K=K(E)$
having $E$ as one of its edges.  Suppose that $\bs{a}_k$ is the vertex
opposite~$E$ in~$K$, so that $E=[\bs{a}_{k_+},\bs{a}_{k_-}]$, as shown in
\cref{fig: Neumann bc}.  A point~$\bs{x}$ lies on~$E$ iff its barycentric
coordinates $(\xi_1,\xi_2,\xi_3)$ with respect to~$K(E)$ satisfy
\[
\xi_k=0,\qquad\xi_{k_+}+\xi_{k_-}=1,\qquad0\le\xi_{k_+}\le1,
\]
so the affine map $[0,1]\to E$ given by
\[
\xi_{k_-}\mapsto\bs{x}=\xi_{k_-}\bs{a}_{k_-}+(1-\xi_{k_-})\bs{a}_{k_+}
\]
provides a parametric representation of~$E$, yielding the formula
\[
g_i=\len(E)\int_0^1\gN\bigl(\bs{x}(\xi_{k_-})\bigr)
    (1-2\xi_{i_-})\,\ud\xi_{k-},
\]
where $\len(E)=|\bs{a}_{k_-}-\bs{a}_{k_+}|$ denotes the length of~$E$.

Consider the simplest case when $\gN$ is constant.  If $i=k$, then
\[
\int_0^1(1-2\xi_{i_-})\,\ud\xi_{k_-}
    =\int_0^1(1-2\xi_{k_-})\,\ud\xi_{k_-}=0.
\]
If $i=k_+$ then $i_-=k$ so
\[
\int_0^1(1-2\xi_{i_-})\,\ud\xi_{k_-}
    =\int_0^1(1-2\xi_k)\,\ud\xi_{k_-}
    =\int_0^1((1-2\times0)\,\ud\xi_{k_-}=1.
\]
If $i=k_-$ then $i_-=k_+$ so
\[
\int_0^1(1-2\xi_{i_-})\,\ud\xi_{k_-}
    =\int_0^1(1-2\xi_{k_+})\,\ud\xi_{k_-}
    =\int_0^1(1-2(1-\xi_{k_-}))\,\ud\xi_{k_-}
    =\int_0^1(2\xi_{k_-}-1)\,\ud\xi_{k_-}=0.
\]
Thus,
\begin{equation}\label{eq: Neumann load vec nonconforming}
g_k=0=g_{k_-}\quad\text{and}\quad g_{k_+}=\len(E)\gN.
\end{equation}
For a general~$\gN$, we can apply the midpoint rule and obtain
\[
g_k\approx 0\approx g_{k_-}\quad\text{and}\quad
g_{k_+}\approx\len(E)\gN(\mathsf{n}_{k_+}).
\]

Next, we consider the element mass matrix for a boundary edge~$E$,
\[
M^E_{ij}=\int_Ec\psi_j\psi_i\quad\text{for $i$, $j\in\{1,2,3\}$.}
\]
Applying the midpoint rule over~$E$, and using the notation of
\cref{fig: Neumann bc},
\[
M^E_{ij}\approx\len(E)c(\mathsf{n}_{k_+})\psi_j(\mathsf{n}_{k_+})
    \psi_i(\mathsf{n}_{k_+})
    =\begin{cases}
    \len(E)c(\mathsf{n}_{k_+}),&\text{if $i=j=k_+$,}\\
    0,&\text{otherwise}
\end{cases}
\]

%%%%%%%%%%%%%%%%%%%%%%%%%%%%%%%%%%%%%%%%%%%%%%%%%%%%%%%%%%%%%%%%%%%%%%%%%%%%%%%
\section{A mixed method}
We consider Darcy flow with pressure~$p$ and fluid flux~$\bs{u}$, so that
\[
\bs{u}= -a\nabla p\quad\text{and}\quad\nabla\cdot\bs{u}=f
    \quad\text{in $\Omega$,}
\]
with the boundary conditions
\[
\text{$p=g_{\uD}$ on $\Gamma_{\uD}$}\qquad\text{and}\qquad
\text{$\bs{n}\cdot\bs{u}=g_{\uN}$ on $\Gamma_{\uN}$.}
\]
For test functions $\bs{v}$~and $w$,
\[
\int_\Omega a^{-1}\bs{u}\cdot\bs{v}=-\int_\Omega(\nabla p)\cdot\bs{v}
    =\int_\Omega p\nabla\cdot\bs{v}
    -\int_{\partial\Omega}p\bs{n}\cdot\bs{v}
\]
and
\[
\int_\Omega(\nabla\cdot\bs{u})w=\int_\Omega fw.
\]
Enforcing the essential boundary condition $\bs{n}\cdot\bs{u}=g_{\uN}$
on~$\Gamma_{\uN}$, and assuming that the test function~$\bs{v}$ satisfies
$\bs{n}\cdot\bs{v}=0$ on~$\Gamma_{\uN}$, we have
\begin{align*}
\int_{\Omega}a^{-1}\bs{u}\cdot\bs{v}-\int_\Omega p\,(\nabla\cdot\bs{v})
    &=-\int_{\Gamma_{\uD}}g_{\uD}\bs{n}\cdot\bs{v},\\
-\int_\Omega(\nabla\cdot\bs{u})w&=\int_\Omega fw.
\end{align*}

For a given triangulation of~$\Omega$, let $\mathrm{RT}_h$ be the space of
lowest-order Raviart--Thomas elements, and $W_h$ the space of piecewise
constants.  The $\mathrm{RT}_h$ shape functions on an element with vertices
$\bs{a}_1$, $\bs{a}_2$, $\bs{a}_3$ are of the form
\[
\bs{\psi}_j(\bs{x})=C_j(\bs{x}-\bs{a}_j)\quad\text{for $j\in\{1,2,3\}$.}
\]
For each edge~$E$ in the mesh we choose a unit normal~$\bs{n}^E$

%%%%%%%%%%%%%%%%%%%%%%%%%%%%%%%%%%%%%%%%%%%%%%%%%%%%%%%%%%%%%%%%%%%%%%%%%%%%%%%
\section{Linear elasticity}
We consider an elastic material with a displacement field
$\bs{u}=\bs{u}(\bs{x})$ and strain tensor
\[
\bs{\varepsilon}=\bs{\varepsilon}[\bs{u}]
    =\tfrac12(\nabla\bs{u}+\nabla\bs{u}^\top),
\]
where
\[
(\nabla\bs{u})_{pq}=\frac{\partial u_p}{\partial x_q}=\partial_qu_p
\quad\text{and}\quad
(\nabla\bs{u})^\top_{pq}=(\nabla\bs{u})_{qp},
\]
so
\[
\varepsilon_{pq}=\tfrac12(\partial_qu_p+\partial_pu_q)=\varepsilon_{qp}.
\]
The stress--strain relation is assumed to be of the form
\begin{equation}\label{eq: stress-strain}
\sigma_{pq}=\lambda\delta_{pq}\varepsilon_{rr}+2\mu\varepsilon_{pq},
\end{equation}
where the Lam\'e parameters $\lambda$~and $\mu$ are positive. The equation of
motion is then
\[
\rho\ddot{\bs{u}}-\nabla\cdot\bs{\sigma}=\bs{f}(\bs{x},t)
    \quad\text{for $\bs{x}\in\Omega$ and $t\ge0$,}
\]
assuming a body force~$\bs{f}$.  We will consider only the steady-state case,
\begin{equation}\label{eq: linear elasticity a}
-\nabla\cdot\bs{\sigma}=f(\bs{x})\quad\text{for $\bs{x}\in\Omega$,}
\end{equation}
with boundary conditions
\begin{equation}\label{eq: linear elasticity b}
\begin{aligned}
\bs{u}&=\bs{g}_{\mathrm{D}}(\bs{x})&&\text{for $\bs{x}\in\GammaD$,}\\
\bs{n}\cdot\bs{\sigma}&=\bs{g}_{\mathrm{N}}(\bs{x})&&\text{for
$\bs{x}\in\GammaN$.}
\end{aligned}
\end{equation}
Since $\varepsilon_{rr}=\partial_ru_r=\nabla\cdot\bs{u}$, if $\lambda$ and
$\mu$ are constant we find that
\[
\nabla\cdot\bs{\sigma}=\mu\nabla^2\bs{u}+(\lambda+\mu)\nabla(\nabla\cdot\bs{u}),
\]
whereas for variable $\lambda$ and $\mu$,
\[
\nabla\cdot\bs{\sigma}=\nabla(\lambda\nabla\cdot\bs{u})
    +\nabla\cdot(\mu\nabla\bs{u})+\nabla\cdot\bigl(\mu(\nabla\bs{u})^\top\bigr).
\]

To derive the weak formulation
of~\eqref{eq: linear elasticity a}--\eqref{eq: linear elasticity b}, we
introduce a vector test function~$\bs{v}$ and apply the divergence theorem as
follows:
\begin{align*}
\int_\Omega(\nabla\cdot\bs{\sigma})\cdot\bs{v}
    &=\int_\Omega(\partial_p\sigma_{pq})v_q
    =\int_\Omega\bigl[\partial_p(\sigma_{pq}v_q)-\sigma_{pq}\partial_pv_q\bigr]
=\int_{\partial\Omega}n_p\sigma_{pq}v_q-\int_\Omega\sigma_{pq}\partial_pv_q\\
    &=\int_{\partial\Omega}(\bs{n}\cdot\bs{\sigma})\cdot\bs{v}
    -\int_\Omega\bs{\sigma}\mathbin{:}\nabla\bs{v}.
\end{align*}
Since
\[
\bs{\sigma}\mathbin{:}\nabla\bs{v}=\lambda(\nabla\cdot\bs{u})(\nabla\cdot\bs{v})
    +2\mu\bs{\varepsilon}[\bs{u}]\mathbin{:}\bs{\varepsilon}[\bs{v}],
\]
the weak solution~$\bs{u}$ must satisfy $\bs{u}=\bs{g}_{\uD}$ on~$\GammaD$, and
\[
\int_\Omega\Bigl(\lambda(\nabla\cdot\bs{u})(\nabla\cdot\bs{v})
    +2\mu\bs{\varepsilon}[\bs{u}]\mathbin{:}\bs{\varepsilon}[\bs{v}]\Bigr)
    =\int_\Omega\bs{f}\cdot\bs{v}+\int_{\GammaN}\bs{g}_{\uN}\cdot\bs{v}
\quad\text{whenever $\bs{v}=\bs{0}$ on $\GammaD$.}
\]

In a 2D problem, we have
\[
\bs{u}(\bs{x})=\begin{bmatrix}u_1(\bs{x})\\ u_2(\bs{x})\end{bmatrix}
    \approx\bs{u}_h(\bs{x})
    =\begin{bmatrix}u_{1h}(\bs{x})\\ u_{2h}(\bs{x})\end{bmatrix}
\]
Consider a triangular element~$K$ with vertices $\mathsf{n}_1$, $\mathsf{n}_2$,
$\mathsf{n}_3$ and scalar linear shape functions 
\[
\psi_j(\bs{x})=\xi_j\quad\text{for $j\in\{1,2,3\}$.}
\]
We define the vector shape functions
\begin{equation}\label{eq: vector shape}
\bs{\psi}_1=\psi_1\,\bs{e}_1,\quad
\bs{\psi}_2=\psi_2\,\bs{e}_1,\quad
\bs{\psi}_3=\psi_3\,\bs{e}_1,\quad
\bs{\psi}_4=\psi_1\,\bs{e}_2,\quad
\bs{\psi}_5=\psi_2\,\bs{e}_2,\quad
\bs{\psi}_6=\psi_3\,\bs{e}_2,
\end{equation}
and the corresponding $6\times6$ element matrices
\[
A_{ij}=\int_K\lambda(\nabla\cdot\bs{\psi}_j)(\nabla\cdot\bs{\psi}_i)
\quad\text{and}\quad
B_{ij}=\int_K2\mu\varepsilon[\bs{\psi}_j]:\varepsilon[\bs{\psi}_i].
\]
Recall that
\[
\nabla\psi_j(\bs{x})=\bs{b}_j=\begin{bmatrix}b_{1j}\\ b_{2j}\end{bmatrix}
\quad\text{so}\quad\partial_p\psi_j=b_{pj},
\]
so
\[
\nabla\cdot\bs{\psi}_j=\partial_1\psi_j=b_{1j}
\quad\text{and}\quad
\nabla\cdot\bs{\psi}_{j+3}=\partial_2\psi_j=b_{2j}
\quad\text{for $1\le j\le3$.}
\]
Thus, for $i$, $j\in\{1,2,3\}$,
\[
\begin{aligned}
A_{ij}&=b_{1j}b_{1i}\int_K\lambda,&\quad 
A_{i,j+3}&=b_{2j}b_{1i}\int_K\lambda,\\
A_{i+3,j}&=b_{1j}b_{2i}\int_K\lambda,&\quad 
A_{i+3,j+3}&=b_{2j}b_{2i}\int_K\lambda.
\end{aligned}
\]

Since
\[
\nabla\bs{\psi}_j=\begin{bmatrix}
\partial_1(\bs{\psi}_j)_1&\partial_2(\bs{\psi}_j)_1 \\           
\partial_1(\bs{\psi}_j)_2&\partial_2(\bs{\psi}_j)_2\end{bmatrix}
\]
we have
\[
\nabla\bs{\psi}_j=\begin{bmatrix}
b_{1j}&b_{2j}\\0&0\end{bmatrix}
\quad\text{and}\quad
\nabla\bs{\psi}_{j+3}=\begin{bmatrix}
0&0\\b_{1j}&b_{2j}\end{bmatrix}
\quad\text{for $1\le j\le 3$,}
\]
so
\[
\bs{\varepsilon}[\bs{\psi}_j]=\begin{bmatrix}
b_{1j}&\tfrac12b_{2j}\\
\tfrac12b_{2j}&0 \end{bmatrix}
\quad\text{and}\quad
\bs{\varepsilon}[\bs{\psi}_{j+3}]=\begin{bmatrix}
0&\tfrac12b_{1j}\\
\tfrac12b_{1j}&b_{2j} \end{bmatrix}\quad\text{for $1\le j\le 3$.}
\]
It follows that
\begin{align*}
\bs{\varepsilon}[\bs{\psi}_j]:\bs{\varepsilon}[\bs{\psi}_i]
    &=b_{1j}b_{1i}+\tfrac14b_{2j}b_{2i}+\tfrac14b_{2j}b_{2i}
    =b_{1j}b_{1i}+\tfrac12 b_{2j}b_{2i},\\
\bs{\varepsilon}[\bs{\psi}_{j+3}]:\bs{\varepsilon}[\bs{\psi}_i]
    &=\tfrac14b_{1j}b_{2i}+\tfrac14b_{1j}b_{2i}=\tfrac12b_{1j}b_{2i},\\
\bs{\varepsilon}[\bs{\psi}_j]:\bs{\varepsilon}[\bs{\psi}_{i+3}]
    &=\tfrac14b_{2j}b_{1i}+\tfrac14b_{2j}b_{1i}=\tfrac12b_{2j}b_{1i},\\
\bs{\varepsilon}[\bs{\psi}_{j+3}]:\bs{\varepsilon}[\bs{\psi}_{i+3}]
    &=\tfrac14b_{1j}b_{1i}+\tfrac14b_{1j}b_{1i}+b_{2j}b_{2i}
    =\tfrac12b_{1j}b_{1i}+b_{2j}b_{2i},
\end{align*}
and hence, for $i$, $j\in\{1,2,3\}$,
\[
\begin{aligned}
B_{ij}&=(2b_{1j}b_{1i}+b_{2j}b_{2i})\int_K\mu,&\quad
B_{i,j+3}&=b_{1j}b_{2i}\int_K\mu,\\
B_{i+3,j}&=b_{2j}b_{1i}\int_K\mu,&\quad
B_{i+3,j+3}&=(b_{1j}b_{1i}+2b_{2j}b_{2i})\int_K\mu.
\end{aligned}
\]

The $6\times6$ element mass matrix is defined by
\begin{equation}\label{eq: elasticity M}
M_{ij}=\int_K c\bs{\psi}_j\cdot\bs{\psi}_i
\end{equation}
We see that if $i$, $j\in\{1,2,3\}$ then
\[
M_{ij}=M_{i+3,j+3}=\int_Kc\psi_j\psi_i
\]
is just the mass matrix for the Poisson problem, and 
\[
M_{i+3,j}=M_{i,j+3}=0.
\]
The components of the $6$-dimensional element load vector are
\begin{equation}\label{eq: elasticity load vec}
f_i=\int_K\bs{f}\cdot{\bs{\psi}_i},
\end{equation}
so if $1\le i\le3$ then
\[
f_i=\int_Kf_1\psi_i\quad\text{and}\quad
f_{i+3}=\int_Kf_2\psi_i,
\]
which also have the same form as for the Poisson problem.

For a boundary edge~$E=[\mathsf{n}^E_1,\mathsf{n}^E_2]$, the associated 
$4$-dimensional load vector is
\[
g_{\uN,i}=\int_E\bs{g}_{\uN}\cdot\bs{\psi}^E_i,
\]
where 
\[
\bs{\psi}^E_1=\psi^E_1\bs{e}_1,\qquad
\bs{\psi}^E_2=\psi^E_2\bs{e}_1,\qquad
\bs{\psi}^E_3=\psi^E_1\bs{e}_2,\qquad
\bs{\psi}^E_4=\psi^E_2\bs{e}_2.
\]
Thus, if $1\le i\le2$ then
\[
g_{\uN,i}=\int_Kg_{\uN,1}\psi^E_i\quad\text{and}\quad
g_{\uN,i+3}=\int_Kg_{\uN,2}\psi^E_i.
\]
In particular, since $\int_E\psi^E_i=\tfrac12\len(E)$, if $\bs{g}_{\uN}$ is a 
constant vector, then
\[
g_{\uN,i}=\tfrac12\len(E)g_{\uN,1}
\quad\text{and}\quad
g_{\uN,i+3}=\tfrac12\len(E)g_{\uN,2}.
\]
For a variable~$\bs{g}_{\uN}$, we recall \eqref{eq: boundary element vector}
and use
\[
g_{\uN,i}\approx\sum_{j=1}^2M^E_{ij}g_{\uN,1}(\bs{a}_j)
\quad\text{and}\quad
g_{\uN,i+2}\approx\sum_{j=1}^2M^E_{ij}g_{\uN,2}(\bs{a}_j).
\]

Let the free nodes be $\mathsf{n}_r$ for $1\le r\le M\free$, and fixed nodes 
(on $\GammaD$) be $\mathsf{n}_r$ for $M\free+1\le r\le M=M\free+M\fix$.  Denote 
the nodal basis function associated with~$\mathsf{n}_r$ by~$\chi_r$ so that
$\chi_r(\mathsf{n}_s)=\delta_{rs}$, and put
\[
\begin{bmatrix}U_{1r}\\ U_{2r}\end{bmatrix}=\bs{u}_h(\mathsf{n}_r)
\quad\text{for $1\le r\le M$.}
\]
We define the vectors
\[
\bs{U}_p\free=[U_{pr}]_{1\le r\le M\free}
\quad\text{and}\quad
\bs{U}_p\fix=[U_{p,M\free+r}]_{1\le r\le M\fix}\quad\text{for $p\in\{1,2\}$,}
\]
and combine these to form the complete nodal vector
\[
\bs{U}=\left[\begin{array}{c}
\bs{U}_1\free\\ \bs{U}_2\free\\ 
\bs{U}_1\fix\\ \bs{U}_2\fix\end{array}\right].
\]
The matrix assembly process is organised to produce the global stiffness matrix
\[
\renewcommand{\arraystretch}{1.2}
\bs{S}=\left[\begin{array}{c|c|c|c}
\bs{S}_{11}\free&\bs{S}_{12}\free&\bs{S}_{11}\fix&\bs{S}_{12}\fix\\
\hline
\bs{S}_{21}\free&\bs{S}_{22}\free&\bs{S}_{21}\fix&\bs{S}_{22}\fix
\end{array}\right].
\]
To satisfy the essential boundary condition, we must put
\[
\bs{U}\fix_p=\bs{g}_{\uD,p}\fix
    \equiv[\bs{g}_{\uD}(\mathsf{n}_{M\free+r})]_{1\le r\le M\fix}
\quad\text{for $p\in\{1,2\}$,}
\]
and then require
\[
\bs{S}\bs{U}=\bs{f}\free+\bs{g}_{\uN}\free,
\]
that is, after eliminating $\bs{U}\fix$,
\[
\renewcommand{\arraystretch}{1.2}
\left[\begin{array}{c|c}
\bs{S}_{11}\free&\bs{S}_{12}\free\\ \hline
\bs{S}_{21}\free&\bs{S}_{22}\free\end{array}\right]
\left[\begin{array}{c}\bs{U}_1\free\\ \hline \bs{U}_2\free\end{array}\right]
=\left[\begin{array}{c}\bs{f}_1\free\\ \hline \bs{f}_2\free\end{array}\right]
+\left[\begin{array}{c}\bs{g}_{\uN,1}\free\\ \hline 
    \bs{g}_{\uN,2}\free\end{array}\right]
    -\left[\begin{array}{c|c}
\bs{S}_{11}\fix&\bs{S}_{12}\fix\\ \hline
\bs{S}_{21}\fix&\bs{S}_{22}\fix\end{array}\right]
\left[\begin{array}{c}\bs{g}_{\uD,1}\fix\\ \hline 
    \bs{g}_{\uD,2}\fix\end{array}\right]
\]

\section{Another formulation for linear elasticity}

Since
\[
\partial_p\sigma_{pq}=\partial_p(\lambda\delta_{pq}\varepsilon_{rr}
    +2\mu\varepsilon_{pq})
    =\partial_q(\lambda\partial_ru_r)
    +\partial_p(\mu\partial_pu_q+\mu\partial_qu_p)
\]
and
\[
\partial_p(\mu\partial_qu_p)
    =(\partial_p\mu)(\partial_qu_p)+\mu\partial_p(\partial_qu_p)
    =(\partial_p\mu)(\partial_qu_p)+\mu\partial_q(\partial_pu_p),
\]
we have
\[
\partial_p\sigma_{pq}=\partial_q(\lambda\partial_pu_p)
    +\partial_p(\mu\partial_pu_q)+(\partial_p\mu)(\partial_qu_p)
    +\mu\partial_q(\partial_pu_p),
\]
or in other words,
\[
\nabla\cdot\bs{\sigma}=\nabla(\lambda\nabla\cdot\bs{u})
    +\nabla\cdot(\mu\nabla\bs{u})+(\nabla\mu)\cdot(\nabla\bs{u})^\top
    +\mu\nabla(\nabla\cdot\bs{u}).
\]
Furthermore, $\partial_q(\mu\partial_pu_p)=(\partial_q\mu)(\partial_pu_p)
+\mu\partial_q\partial_pu_p$ so
\[
\nabla(\mu\nabla\cdot\bs{u})=(\nabla\mu)(\nabla\cdot\bs{u})
    +\mu\nabla(\nabla\cdot\bs{u})
\]
and hence
\begin{align*}
\nabla\cdot\bs{\sigma}&=\nabla(\lambda\nabla\cdot\bs{u})
    +\nabla\cdot(\mu\nabla\bs{u})+(\nabla\mu)\cdot(\nabla\bs{u})^\top
    +\nabla(\mu\nabla\cdot\bs{u})-(\nabla\mu)(\nabla\cdot\bs{u})\\
&=\nabla\cdot(\mu\nabla\bs{u})+\nabla\bigl((\mu+\lambda)\nabla\cdot\bs{u}\bigr)
    +\nabla\mu\cdot(\nabla\bs{u})^\top-(\nabla\mu)(\nabla\cdot\bs{u}).
\end{align*}
Taking the dot product of the first two terms with a test function $\bs{v}$, we
find that
\begin{align*}
\bigl[\nabla\cdot(\mu\nabla\bs{u})
    &+\nabla\bigl((\mu+\lambda)\nabla\cdot\bs{u}\bigr)\bigr]\cdot\bs{v}
    =\sum_{p=1}^2\sum_{q=1}^2\bigl[\partial_p(\mu\partial_pu_q)
    +\partial_q\bigl((\mu+\lambda)\partial_pu_p\bigr)\bigr]v_q\\
    &=\sum_{p=1}^2\sum_{q=1}^2\bigl[\partial_p\bigl(
    \mu(\partial_pu_q)v_q+(\mu+\lambda)(\partial_pu_p)v_q\bigr)
    -\mu(\partial_pu_q)(\partial_pv_q)
    -(\mu+\lambda)(\partial_qv_q)(\partial_pv_p)\bigr]\\
    &=\nabla\cdot\bigl(\mu(\nabla\bs{u})\cdot\bs{v}
    +(\mu+\lambda)(\nabla\cdot\bs{u})\bs{v})\bigr)
    -\mu\nabla\bs{u}:\nabla\bs{v}
    -(\mu+\lambda)(\nabla\cdot\bs{u})(\nabla\cdot\bs{v}),
\end{align*}
and for the last two terms,
\begin{align*}
\bigl[(\nabla\mu)\cdot(\nabla\bs{u})^\top&-(\nabla\mu)(\nabla\cdot\bs{u})\bigr]
    \cdot\bs{v}
=\sum_{p=1}^2\sum_{q=1}^2\bigl(\partial_p\mu\,\partial_qu_p
    -\partial_q\mu\, \partial_pu_p\bigr)v_q\\
&=\sum_{p=1}^2
    \bigl(\partial_p\mu\,\partial_1u_p-\partial_1\mu\,\partial_pu_p\bigr)v_1
+\sum_{p=1}^2
    \bigl(\partial_p\mu\,\partial_2u_p-\partial_2\mu\,\partial_pu_p\bigr)v_2\\
&=\bigl(\partial_2\mu\,\partial_1u_2-\partial_1\mu\,\partial_2u_2\bigr)v_1
+\bigl(\partial_1\mu\,\partial_2u_1-\partial_2\mu\,\partial_1u_1\bigr)v_2\\
&=-\begin{bmatrix}-\partial_2\mu\\ \partial_1\mu\end{bmatrix}\cdot
\begin{bmatrix}\partial_1u_2\\ \partial_2u_2 \end{bmatrix}v_1
-\begin{bmatrix}\partial_1\mu\\ \partial_2\mu \end{bmatrix}\cdot
\begin{bmatrix}-\partial_2u_1\\ \partial_1u_1 \end{bmatrix}v_2,
\end{align*}
or in other words,
\[
\bigl[\nabla\mu\cdot(\nabla\bs{u})^\top-(\nabla\mu)(\nabla\cdot\bs{u})\bigr]
    \cdot\bs{v}
=-(\nabla\mu)\rot\cdot(\nabla u_2)\,v_1
    -(\nabla\mu)\cdot(\nabla u_1)\rot\,v_2,
\]
where we have used the notation
\[
\bs{a}\rot=\begin{bmatrix}-a_2\\ a_1 \end{bmatrix}
\quad\text{for $\bs{a}\in\mathbb{R}^2$.}
\]
Thus, $\bs{a}\rot$ is the vector obtained by rotating~$\bs{a}$
counterclockwise through a right angle.

It follows that
\begin{multline*}
-(\nabla\cdot\bs{\sigma})\cdot\bs{v}=\mu\nabla\bs{u}:\nabla\bs{v}
    +(\mu+\lambda)(\nabla\cdot\bs{u})(\nabla\cdot\bs{v})
    +(\nabla\mu)\rot\cdot(\nabla u_2)\,v_1
    +(\nabla\mu)\cdot(\nabla u_1)\rot\,v_2\\
    -\nabla\cdot\bigl(\mu(\nabla\bs{u})\cdot\bs{v}
    +(\mu+\lambda)(\nabla\cdot\bs{u})\bs{v})\bigr)
\end{multline*}
and hence
\begin{multline}\label{eq: alt elasticity}
\int_\Omega(-\nabla\cdot\bs{\sigma})\cdot\bs{v}=\int_\Omega\Bigl(
    \mu\nabla\bs{u}:\nabla\bs{v}+(\mu+\lambda)
        (\nabla\cdot\bs{u})(\nabla\cdot\bs{v})
    +(\nabla\mu)\rot\cdot(\nabla u_2)\,v_1
    +(\nabla\mu)\cdot(\nabla u_1)\rot\,v_2\Bigr)\\
    -\int_{\partial\Omega}\Bigl(\mu\bs{n}\cdot(\nabla\bs{u})\cdot\bs{v}
    +(\mu+\lambda)(\nabla\cdot\bs{u})(\bs{n}\cdot\bs{v})\Bigr).
\end{multline}
Since
\[
\nabla\cdot\bigl(u_2v_1(\nabla\mu)\rot\bigr)
    =u_2\nabla\cdot\bigl(v_1(\nabla\mu)\rot\bigr)
    +(\nabla u_2)\cdot\bigl(v_1(\nabla\mu)\rot\bigr)
\]
we have
\[
(\nabla\mu)\rot\cdot(\nabla u_2)\,v_1
    =\nabla\cdot\bigl(u_2v_1(\nabla\mu)\rot\bigr)
    -u_2\nabla\cdot\bigl(v_1(\nabla\mu)\rot\bigr).
\]
Furthermore,
\[
\nabla\cdot\bigl(v_1(\nabla\mu)\rot\bigr)
    =(\nabla v_1)\cdot(\nabla\mu)\rot
    +v_1\nabla\cdot(\nabla\mu)\rot
\]
and
\[
\nabla\cdot(\nabla\mu)\rot=\partial_1(-\partial_2\mu)
    +\partial_2(\partial_1\mu)=0.
\]
Therefore, noting that $\bs{a}\cdot\bs{b}\rot=-\bs{a}\rot\cdot\bs{b}$,
\[
(\nabla\mu)\rot\cdot(\nabla u_2)\,v_1
    =\nabla\cdot\bigl(u_2v_1(\nabla\mu)\rot\bigr)
    -u_2(\nabla v_1)\cdot(\nabla\mu)\rot
    =\nabla\cdot\bigl(u_2v_1(\nabla\mu)\rot\bigr)
    +u_2(\nabla\mu)\cdot(\nabla v_1)\rot,
\]
so
\begin{multline}\label{eq: variable mu}
\int_\Omega(-\nabla\cdot\bs{\sigma})\cdot\bs{v}=\int_\Omega\Bigl(
    \mu\nabla\bs{u}:\nabla\bs{v}+(\mu+\lambda)
        (\nabla\cdot\bs{u})(\nabla\cdot\bs{v})
    +(\nabla\mu)\cdot[(\nabla u_1)\rot\,v_2
    +u_2(\nabla v_1)\rot]\Bigr)\\
    -\int_{\partial\Omega}\Bigl(\mu\bs{n}\cdot(\nabla\bs{u})\cdot\bs{v}
    +(\mu+\lambda)(\nabla\cdot\bs{u})(\bs{n}\cdot\bs{v})
    -u_2v_1\,\bs{n}\cdot(\nabla\mu)\rot\Bigr),
\end{multline}
Notice that the integral over~$\partial\Omega$ vanishes if $\bs{v}\equiv0$
on~$\partial\Omega$, which will be the case for a pure Dirichlet problem.

For an alternative approach, we observe that
\[
2\bs{\varepsilon}[\bs{u}]:\bs{\varepsilon}[\bs{v}]
=\frac12\sum_{p=1}^2\sum_{q=1}^2(\partial_pu_q+\partial_qu_p)
(\partial_pv_q+\partial_qv_p)
\]
and, by swapping dummy indices,
\[
\sum_{p=1}^2\sum_{q=1}^2\partial_pu_q\,\partial_pv_q
    =\sum_{p=1}^2\sum_{q=1}^2\partial_qu_p\,\partial_qv_p
    =\nabla\bs{u}:\nabla\bs{v}
\]
with
\[
\sum_{p=1}^2\sum_{q=1}^2\partial_pu_q\,\partial_qv_p
    =\sum_{p=1}^2\sum_{q=1}^2\partial_qu_p\,\partial_pv_q.
\]
Thus,
\[
2\bs{\varepsilon}[\bs{u}]:\bs{\varepsilon}[\bs{v}]-\nabla\bs{u}:\nabla\bs{v}
    =\sum_{p=1}^2\sum_{q=1}^2\partial_pu_q\,\partial_qv_p,
\]
whereas
\[
(\nabla\cdot\bs{u})(\nabla\cdot\bs{v})=\sum_{p=1}^2\sum_{q=1}^2
    \partial_pu_p\,\partial_qv_q,
\]
so
\begin{align*}
2\bs{\varepsilon}[\bs{u}]:\bs{\varepsilon}[\bs{v}]&-\nabla\bs{u}:\nabla\bs{v}
    -(\nabla\cdot\bs{u})(\nabla\cdot\bs{v})
    =(\partial_1u_2\,\partial_2v_1+\partial_2u_1\,\partial_1v_2)
    -(\partial_1u_1\,\partial_2v_2+\partial_2u_2\,\partial_1v_1)\\
    &=-[(-\partial_2u_1)(\partial_1v_2)+(\partial_1u_1)(\partial_2v_2)]
    -[(\partial_1u_2)(-\partial_2v_1)+(\partial_2u_2)(\partial_1v_1)]\\
    &=-(\nabla u_1)\rot\cdot(\nabla v_2)
        -(\nabla u_2)\cdot(\nabla v_1)\rot.
\end{align*}
It follows that
\begin{align*}
\lambda(\nabla\cdot\bs{u})(\nabla\cdot\bs{v})
+2\mu\bs{\varepsilon}[\bs{u}]:\bs{\varepsilon}[\bs{v}]
    &=\mu\nabla\bs{u}:\nabla\bs{v}
    +(\lambda+\mu)(\nabla\cdot\bs{u})(\nabla\cdot\bs{v})\\
    &\qquad{}-\mu[(\nabla u_1)\rot\cdot(\nabla v_2)
        +(\nabla u_2)\cdot(\nabla v_1)\rot)].
\end{align*}
Since $\nabla\cdot(\nabla u_1)\rot
=\partial_1(-\partial_2u_1)+\partial_2(\partial_1u_1)=0$,
\[
\nabla\cdot\bigl(\mu v_2(\nabla u_1)\rot\bigr)
    =\nabla(\mu v_2)\cdot(\nabla u_1)\rot+\mu v_2\nabla\cdot(\nabla u_1)\rot
    =\mu(\nabla v_2)\cdot(\nabla u_1)\rot+v_2(\nabla\mu)\cdot(\nabla u_1)\rot
\]
and thus
\[
\mu(\nabla u_1)\rot\cdot(\nabla v_2)
    =\nabla\cdot\bigl(\mu(\nabla u_1)\rot v_2\bigr)
    -(\nabla\mu)\cdot((\nabla u_1)\rot v_2\bigr).
\]
Similarly,
\[
\mu(\nabla u_2)\cdot(\nabla v_1)\rot
    =\nabla\cdot\bigl(\mu u_2(\nabla v_1)\rot\bigr)
    -(\nabla\mu)\cdot\bigl(u_2(\nabla v_1)\rot\bigr),
\]
implying that
\[
\mu[(\nabla u_1)\rot\cdot(\nabla v_2)+(\nabla u_2)\cdot(\nabla v_1)\rot)]
=\nabla\cdot\bigl[\mu(\nabla u_1)\rot v_2+\mu u_2(\nabla v_1)\rot\bigr]
-(\nabla\mu)\cdot\bigl[(\nabla u_1)\rot v_2+u_2(\nabla v_1)\rot]
\]
and so
\begin{multline*}
-\int_\Omega\mu[(\nabla u_1)\rot\cdot(\nabla v_2)
        +(\nabla u_2)\cdot(\nabla v_1)\rot)]
    =\int_\Omega(\nabla\mu)\cdot\bigl[(\nabla u_1)\rot v_2
    +u_2(\nabla v_1)\rot]\\
    -\int_{\partial\Omega}\mu\bigl[
    \bs{n}\cdot(\nabla u_1)\rot v_2+u_2\,\bs{n}\cdot(\nabla v_1)\rot\bigr],
\end{multline*}
Thus,
\begin{align*}
\int_\Omega\Bigl(\lambda(\nabla\cdot\bs{u})(\nabla\cdot\bs{v})
&+2\mu\bs{\varepsilon}[\bs{u}]:\bs{\varepsilon}[\bs{v}]\Bigr)\\
    &=\int_\Omega\Bigl(\mu\nabla\bs{u}:\nabla\bs{v}
    +(\lambda+\mu)(\nabla\cdot\bs{u})(\nabla\cdot\bs{v})
    +(\nabla\mu)\cdot\bigl[(\nabla u_1)\rot v_2+u_2(\nabla v_1)\rot]\Bigr)\\
    &\qquad{}-\int_{\partial\Omega}\mu\bigl[
    \bs{n}\cdot(\nabla u_1)\rot v_1+u_2\,\bs{n}\cdot(\nabla v_1)\rot\bigr].
\end{align*}

%%%%%%%%%%%%%%%%%%%%%%%%%%%%%%%%%%%%%%%%%%%%%%%%%%%%%%%%%%%%%%%%%%%%%%%%%%%%%%%
\section{Nonconforming method for linear elasticity}
Assume a pure Dirichlet problem so that, by~\eqref{eq: alt elasticity},
\[
\int_\Omega\Bigl(\mu\nabla\bs{u}:\nabla\bs{v}+(\mu+\lambda)
        (\nabla\cdot\bs{u})(\nabla\cdot\bs{v})
    +(\nabla\mu)\cdot[(\nabla u_1)\rot\,v_2
    +u_2(\nabla v_1)\rot]\Bigr)=\int_\Omega\bs{f}\cdot\bs{v}
\]
when~$\bs{v}=\bs{0}$ on~$\partial\Omega$.  The six vector shape functions are 
again given by~\eqref{eq: vector shape}, except that we now use the 
nonconforming $\psi_1$, $\psi_2$, $\psi_3$ given 
by~\eqref{eq: psi_i nonconforming}, and the nodes $\mathsf{n}_1$, 
$\mathsf{n}_2$, $\mathsf{n}_3$ are now the midpoints of the triangle~$K$.

Define the $6\times6$ element matrices
\[
A_{ij}=\int_K(\mu+\lambda)(\nabla\cdot\bs{\psi}_j)(\nabla\cdot\bs{\psi}_i)
\quad\text{and}\quad
B_{ij}=\int_K\mu\nabla\bs{\psi}_j:\nabla\bs{\psi}_i,
\]
and recall from~\eqref{eq: grad psi_i} that $\nabla\psi_i=-2\bs{b}_{i_-}$, so
\[
\nabla\bs{\psi}_j=\begin{bmatrix}
-2b_{1j_-}&-2b_{2j_-}\\ 0&0\end{bmatrix}
\quad\text{and}\quad
\nabla\bs{\psi}_{j+3}=\begin{bmatrix}
0&0\\ -2b_{1j_-}&-2b_{2j_-}\end{bmatrix}
\quad\text{for $1\le j\le3$.}
\]
Similarly,
\[
\nabla\cdot\bs{\psi}_j=\partial_1\psi_j=-2b_{1,j_-}
\quad\text{and}\quad
\nabla\cdot\bs{\psi}_{j+3}=\partial_2\psi_j=-2b_{2,j_-}
\quad\text{for $1\le j\le3$,}
\]
and therefore, if $i$, $j\in\{1,2,3\}$, then
\[
\begin{aligned}
A_{ij}&=4b_{1j_-}b_{1i_-}\int_K(\mu+\lambda),&\quad
A_{i,j+3}&=4b_{2j_-}b_{1i_-}\int_K(\mu+\lambda),\\
A_{i+3,j}&=4b_{1j_-}b_{2i_-}\int_K(\mu+\lambda),&\quad
A_{i+3,j+3}&=4b_{2j_-}b_{2i_-}\int_K(\mu+\lambda), 
\end{aligned}
\]
with
\[
\begin{aligned}
B_{i,j}&=4(b_{1j_-}b_{1i_-}+b_{2j_-}b_{2i_-})\int_K\mu,&\quad
B_{i+3,j}&=0,\\
B_{i,j+3}&=0,&\quad
B_{i+3,j+3}&=B_{ij}.
\end{aligned}
\]

It remains to consider the contribution from
\begin{align*}
D_{ij}&=\int_K(\nabla\mu)\cdot\bigl[
     \bigl(\nabla(\bs{\psi}_j)_1\bigr)\rot(\bs{\psi}_i)_2
    +(\bs{\psi}_j)_2\bigl(\nabla(\bs{\psi}_i)_1\bigr)\rot\bigr]\\
    &=\int_K(\partial_2\mu)\bigl[\partial_1(\bs{\psi}_j)_1(\bs{\psi}_i)_2
    +(\bs{\psi}_j)_2\partial_1(\bs{\psi}_i)_1\bigr]\\
    &\qquad{}-\int_K(\partial_1\mu)\bigl[
    \partial_2(\bs{\psi}_j)_1(\bs{\psi}_i)_2
    +(\bs{\psi}_j)_2\partial_2(\bs{\psi}_i)_1\bigr].
\end{align*}
If $i$, $j\in\{1,2,3\}$, then
\[
D_{ij}=0=D_{i+3,j+3}
\]
with
\begin{align*}
D_{i,j+3}&=\int_K(\partial_2\mu)\bigl[0+\psi_j\partial_1\psi_i\bigr]
    -\int_K(\partial_1\mu)\bigl[0+\psi_j\partial_2\psi_i\bigr]\\
    &=\int_K\bigl[(\partial_2\mu)(-2b_{1i_-})-(\partial_1\mu)(-2b_{2i_-})\bigr]
    \psi_j\\
    &=2\int_K\bigl(b_{2i_-}\partial_1\mu-b_{1i_-}\partial_2\mu\bigr)\psi_j
\end{align*}
and
\begin{align*}
D_{i+3,j}&=\int_K(\partial_2\mu)\bigl[(\partial_1\psi_j)\psi_i+0\bigr]
    -\int_K(\partial_1\mu)\bigl[(\partial_2\psi_j)\psi_i+0\bigr]\\
    &=\int_K\bigl[(\partial_2\mu)(-2b_{1j_-})-(\partial_1\mu)(-2b_{2j_-})\bigr]
    \psi_i\\
    &=2\int_K\bigl(b_{2j_-}\partial_1\mu-b_{1j_-}\partial_2\mu\bigr)\psi_i
    =D_{j+3,i}.
\end{align*}
Since $\int_K\psi_i=\area(K)/3$, if $\nabla\mu$ is constant on~$K$, then
\[
D_{i,j+3}=\tfrac23\area(K)(b_{2i_-}\partial_1\mu-b_{1i_-}\partial_2\mu).
\]
Otherwise, we use the quadrature approximation
\[
\int_Kf\approx\frac{\area(K)}{3}\sum_{k=1}^3f(\mathsf{n}_k),
\]
and since $\psi_j(\mathsf{n}_k)=\delta_{jk}$,
\[
D_{i,j+3}\approx\tfrac23\area(K)\bigl[b_{2i_-}(\partial_1\mu)(\mathsf{n}_j)
    -b_{1i_-}(\partial_2\mu)(\mathsf{n}_j\bigr].
\]

For the load vector, the formula~\eqref{eq: elasticity load vec} again holds 
but with the nonconforming $\psi_i$. Thus, if $\bs{f}$ is constant on~$K$, then 
we still have
\[
f_i=\frac{\area(K)}{3}\,f_1\quad\text{and}\quad
f_{i+3}=\frac{\area(K)}{3}\,f_2\quad\text{for $1\le \le3$.}
\]
If $\bs{f}$ is not constant, then the quadrature
rule~\eqref{eq: edge midpt rule} yields the approximation
\[
f_i=\frac{\area(K)}{3}\,f_1(\mathsf{n}_i)\quad\text{and}\quad
f_{i+3}=\frac{\area(K)}{3}\,f_2(\mathsf{n}_i)\quad\text{for $1\le \le3$.}
\]


%%%%%%%%%%%%%%%%%%%%%%%%%%%%%%%%%%%%%%%%%%%%%%%%%%%%%%%%%%%%%%%%%%%%%%%%%%%%%%%
\appendix
\section{Fundamental solution for linear elasticity}
We wish to find $\bs{u}$ satisfying
\begin{equation}\label{eq: elasticity point force}
-\mu\nabla^2\bs{u}-(\lambda+\mu)\nabla(\nabla\cdot\bs{u})=\bs{f}\delta(\bs{x})
\quad\text{ $\bs{0}\ne\bs{x}\in\mathbb{R}^2$,}
\end{equation}
for a given constant vector~$\bs{f}$.  Denote the fundamental solution for the
Laplacian by
\[
v(\bs{x})=\frac{1}{2\pi}\,\log\frac{1}{|\bs{x}|}
\]
so that $-\nabla^2v=\delta(\bs{x})$, and put
\[
\bs{w}(\bs{x})=\bs{u}(\bs{x})-\mu^{-1}v(\bs{x})\bs{f}
\]
so that
\[
-\mu\nabla^2\bs{w}-(\lambda+\mu)\nabla(\nabla\cdot\bs{w})
    =(\lambda+\mu)\mu^{-1}\nabla\bigl(\nabla\cdot(v\bs{f})\bigr)
\quad\text{ $\bs{0}\ne\bs{x}\in\mathbb{R}^2$.}
\]
Taking the curl of this equation gives
$-\mu\nabla^2(\nabla\times\bs{w})=\bs{0}$, and since $\nabla\times\bs{w}$
vanishes at infinity we conclude that $\nabla\times\bs{w}=\bs{0}$.  It follows
that $\bs{w}=\nabla\phi$ for some potential~$\phi$, and since
$\nabla^2\bs{w}=\nabla^2(\nabla\phi)=\nabla(\nabla^2\phi)$ and
$\nabla(\nabla\cdot\bs{w})=\nabla(\nabla^2\phi)$ we have
\[
-\mu\nabla^2\bs{w}-(\lambda+\mu)\nabla(\nabla\cdot\bs{w})
    =-\nabla\bigl((\lambda+2\mu)\nabla^2\phi\bigr),
\]
implying that
\begin{equation}\label{eq: poisson phi}
-(\lambda+2\mu)\nabla^2\phi=(\lambda+\mu)\mu^{-1}\nabla\cdot(v\bs{f})
    =(\lambda+\mu)\mu^{-1}\bs{f}\cdot\nabla v.
\end{equation}
The radially symmetric function
\[
\psi(\bs{x})=\frac{r^2}{8\pi}\,(1-\log r),\quad r=|\bs{x}|,
\]
satisfies
\[
\nabla^2\psi=\frac{1}{r}\,\frac{\ud}{\ud r}
    \biggl(r\,\frac{\ud\psi}{\ud r}\biggr)
    =\frac{1}{8\pi r}\,\frac{\ud}{\ud r}\bigl(r^2-2r^2\log r\bigr)
    =v(\bs{x})
\]
and therefore
\[
\bs{f}\cdot\nabla v=\bs{f}\cdot\nabla(\nabla^2\psi)
    =\nabla^2(\bs{f}\cdot\nabla\psi),
\]
so \eqref{eq: poisson phi} is satisfied if we put
\[
\phi(\bs{x})=-\frac{\lambda+\mu}{\mu(\lambda+2\mu)}\,\bs{f}\cdot\nabla\psi.
\]
Since $\nabla(\bs{f}\cdot\nabla\psi)=(\bs{f}\cdot\nabla)\nabla\psi$,
\[
\bs{w}=\nabla\phi=-\frac{\lambda+\mu}{\mu(\lambda+2\mu)}\,
    (\bs{f}\cdot\nabla)\nabla\psi
\]
and
\[
\nabla\psi=\frac{1-2\log r}{8\pi}\,\bs{x},
\]
we have
\begin{align*}
(\bs{f}\cdot\nabla)\nabla\psi&=\frac{1}{8\pi}\,f_p\partial_p
    \bigl((1-2\log r)\bs{x}\bigr)
    =\frac{1}{8\pi}\,f_p\biggl(\frac{-2}{r}\,\frac{x_p}{r}\,\bs{x}
    +(1-2\log r)\bs{e}_p\biggr)\\
    &=\frac{1}{8\pi}(1-2\log r)\bs{f}-\frac{1}{4\pi}\,
        \frac{\bs{f}\cdot\bs{x}}{r}\,\frac{\bs{x}}{r}
\end{align*}
and thus
\begin{align*}
\bs{u}(\bs{x})&=\mu^{-1}v(\bs{x})\bs{f}+\bs{w}(\bs{x})
    =\frac{\bs{f}}{2\pi\mu}\,\log\frac{1}{r}
    -\frac{\lambda+\mu}{\mu(\lambda+2\mu)}\biggl(
    \frac{\bs{f}}{8\pi}+\frac{\bs{f}}{4\pi}\log\frac{1}{r}
    -\frac{1}{4\pi}\,\frac{\bs{f}\cdot\bs{x}}{r}\,\frac{\bs{x}}{r}\biggr)\\
    &=\frac{\bs{f}}{4\pi\mu(\lambda+2\mu)}\bigr(
    2(\lambda+2\mu)-(\lambda+\mu)\bigr)\log\frac{1}{r}
        +\frac{\lambda+\mu}{4\pi\mu(\lambda+2\mu)}\biggl(
    \frac{(\bs{f}\cdot\bs{x})\bs{x}}{r^2}-\frac{\bs{f}}{2}\biggr).
\end{align*}
If $\bs{u}$ is a solution of~\eqref{eq: elasticity point force}, then so is
$\bs{u}+\bs{c}$ for any constant vector~$\bs{c}\in\mathbb{R}^2$. Dropping the
constant term, we arrive at the solution
\[
u_p(\bs{x})=\frac{1}{4\pi\mu(\lambda+2\mu)}\biggl(
    (\lambda+3\mu)\delta_{pq}\log\frac{1}{|\bs{x}|}
    +(\lambda+\mu)\frac{x_px_q}{|\bs{x}|^2}\biggr)f_q.
\]
%%%%%%%%%%%%%%%%%%%%%%%%%%%%%%%%%%%%%%%%%%%%%%%%%%%%%%%%%%%%%%%%%%%%%%%%%%%%%%%
\section{Exact solution for elasticity with variable coefficients}

We consider the elasticity equation on the square $\Omega=(0,\pi)^2$ with
Lam\'e coefficients
\[
\lambda(x,y)=\Lambda(1+\alpha\sin(2x))\quad\text{and}\quad\mu=1+\beta(x+y).
\]

\begin{lemma}\label{lem: elasticity div free soln}
Let
\[
\phi(x,y)=2\beta[2\sin(2x)\sin(2y)-\cos(2x)+\cos(2y)]
    +4\mu(x,y)\sin(2y)[2\cos(2x)-1],\\
\]
The vector fields
\[
\bs{u}(x,y)=\begin{bmatrix}
\bigl(\cos(2x)-1\bigr)\sin(2y)\\[1\jot]
\bigl(1-\cos(2y)\bigr)\sin(2x)\end{bmatrix}
\quad\text{and}\quad
\bs{f}(x,y)=\begin{bmatrix}\phi(x,y)\\ -\phi(y,x)\end{bmatrix}
\]
satisfy
\[
\nabla\cdot\bs{u}=0\quad\text{and}\quad
-\nabla\cdot\bs{\sigma}[\bs{u}]=\bs{f}.
\]
\end{lemma}
\begin{proof}
We have
\begin{align*}
\partial_1u_1&=-2\sin(2x)\sin(2y),\\
\partial_1u_2&=2(1-\cos(2y))\cos(2x),\\
\partial_2u_1&=2(\cos(2x)-1)\cos(2y),\\
\partial_2u_2&=2\sin(2y)\sin(2x),
\end{align*}
so the components of the strain tensor
$\varepsilon_{pq}=\tfrac12(\partial_pu_q+\partial_qu_p)$ are
\[
\varepsilon_{11}=-2\sin(2x)\sin(2y),\qquad
\varepsilon_{12}=\varepsilon_{21}=\cos(2x)-\cos(2y),\qquad
\varepsilon_{22}=2\sin(2y)\sin(2x),
\]
and in particular,
\[
\nabla\cdot\bs{u}=\varepsilon_{11}+\varepsilon_{22}
    =\partial_1u_1+\partial_2u_2=0.
\]
Thus, $\sigma_{pq}=2\mu\varepsilon_{pq}$ and so the components of the stress
tensor are
\[
\sigma_{11}=-4\mu\sin(2x)\sin(2y),\qquad
\sigma_{12}=\sigma_{21}=2\mu[\cos(2x)-\cos(2y)],\qquad
\sigma_{22}=4\mu\sin(2y)\sin(2x).
\]
Since $\partial_1\mu=\beta=\partial_2\mu$ and $\mu(x,y)=\mu(y,x)$,
\begin{align*}
\partial_1\sigma_{11}(x,y)&=-4\beta\sin(2x)\sin(2y)-8\mu\cos(2x)\sin(2y),\\
\partial_2\sigma_{21}(x,y)&=2\beta[\cos(2x)-\cos(2y)]+4\mu\sin(2y),\\
\partial_1\sigma_{12}(x,y)&=2\beta[\cos(2x)-\cos(2y)]-4\mu\sin(2x)
    =-\partial_2\sigma_{21}(y,x),\\
\partial_2\sigma_{22}(x,y)&=4\beta\sin(2y)\sin(2x)+8\mu\cos(2y)\sin(2x)
    =-\partial_1\sigma_{11}(y,x),
\end{align*}
and therefore
\[
f_1(x,y)=-(\partial_1\sigma_{11}+\partial_2\sigma_{21})(x,y)=\phi(x,y)
\]
with
\[
f_2(x,y)=-(\partial_1\sigma_{12}+\partial_2\sigma_{22})(x,y)
=(\partial_2\sigma_{21}+\partial_1\sigma_{11})(y,x)=-f_1(y,x)=-\phi(y,x),
\]
as claimed.
\end{proof}

\begin{lemma}\label{lem: elasticity extra soln}
Put $\tau(x,y)=1+\alpha\sin(2x)$, so that $\lambda(x,y)=\Lambda\tau(x,y)$, and
let
\begin{gather*}
\psi_1(x,y)=\mu(x,y)[2\sin(x)\sin(y)-\cos(x+y)]-\beta\sin(x+y),\\
\psi_2(x,y)=2\beta\cos(x)\sin(y),\qquad
\psi_3(x,y)=\tau(x,y)\cos(x+y).
\end{gather*}
The vector fields
\[
\bs{u}(x,y)=\sin(x)\sin(y)\begin{bmatrix}1\\ 1 \end{bmatrix}
\]
and
\[
\bs{f}(x,y)=\begin{bmatrix}
\psi_1(x,y)-\psi_2(x,y)\\
\psi_1(x,y)-\psi_2(y,x)\end{bmatrix}
-\Lambda\begin{bmatrix}
\psi_3(x,y)+2\alpha\cos(2x)\sin(x+y)\\
\psi_3(x,y)\end{bmatrix}
\]
satisfy $-\nabla\cdot\bs{\sigma}[\bs{u}]=\bs{f}$.
\end{lemma}
\begin{proof}
Since $u_1=u_2$,
\begin{align*}
\partial_1u_1&=\partial_1u_2=\cos(x)\sin(y),\\
\partial_2u_1&=\partial_2u_2=\sin(x)\cos(y),
\end{align*}
and the components of the strain tensor are
\[
\varepsilon_{11}=\cos(x)\sin(y),\qquad
\varepsilon_{12}=\varepsilon_{21}=\tfrac12\sin(x+y),\qquad
\varepsilon_{22}=\sin(x)\cos(y),
\]
so in particular,
\[
\nabla\cdot\bs{u}=\varepsilon_{11}+\varepsilon_{22}=\sin(x+y).
\]
Thus, the components of the stress tensor are
\begin{align*}
\sigma_{11}&=\lambda\nabla\cdot\bs{u}+2\mu\varepsilon_{11}
    =\lambda\sin(x+y)+2\mu\cos(x)\sin(y),\\
\sigma_{12}&=2\mu\varepsilon_{12}=\mu\sin(x+y)=\sigma_{21},\\
\sigma_{22}&=\lambda\nabla\cdot\bs{u}+2\mu\varepsilon_{22}
    =\lambda\sin(x+y)+2\mu\sin(x)\cos(y).
\end{align*}
We have
\begin{align*}
\partial_1\sigma_{11}&=(\partial_1\lambda)\sin(x+y)+\lambda\cos(x+y)
    +2(\partial_1\mu)\cos(x)\sin(y)-2\mu\sin(x)\sin(y)\\
    &=\Lambda[(\partial_1\tau)\sin(x+y)+\tau\cos(x+y)]
    +2\beta\cos(x)\sin(y)-2\mu\sin(x)\sin(y),
\end{align*}
with
\[
\partial_2\sigma_{21}=\beta\sin(x+y)+\mu\cos(x+y)=\partial_1\sigma_{12}
\]
and
\[
\partial_2\sigma_{22}=\Lambda\tau\cos(x+y)+2\beta\sin(x)\cos(y)
    -2\mu\sin(x)\sin(y).
\]
Therefore, noting that $\partial_1\tau=2\alpha\cos(2x)$,
\begin{align*}
-f_1&=\partial_1\sigma_{11}+\partial_2\sigma_{21}
    =\Lambda[2\alpha\cos(2x)\sin(x+y)+\tau\cos(x+y)]\\
    &\qquad{}+\beta[2\cos(x)\sin(y)+\sin(x+y)]+\mu[\cos(x+y)-2\sin(x)\sin(y)]\\
    &=\Lambda[\psi_3(x,y)+2\alpha\cos(2x)\sin(x+y)]-\psi_1(x,y)+\psi_2(x,y)
\end{align*}
and
\begin{align*}
-f_2&=\partial_1\sigma_{12}+\partial_2\sigma_{22}\\
    &=\Lambda\tau\cos(x+y)+\beta[2\sin(x)\cos(y)
    +\sin(x+y)]+\mu[\cos(x+y)-2\sin(x)\sin(y)]\\
    &=\Lambda\psi_3(x,y)+\psi_2(y,x)-\psi_1(x,y),
\end{align*}
which implies the result.
\end{proof}

We now use superposition, taking the solution in
\cref{lem: elasticity div free soln} and adding $\Lambda^{-1}$ times the
solution in \cref{lem: elasticity extra soln}, to obtain
\[
\bs{u}(x,y)=\begin{bmatrix}
\bigl(\cos(2x)-1\bigr)\sin(2y)\\[1\jot]
\bigl(1-\cos(2y)\bigr)\sin(2x)\end{bmatrix}
+\frac{\sin(x)\sin(y)}{\Lambda}\begin{bmatrix}1\\ 1 \end{bmatrix}
\]
and
\[
\bs{f}=\begin{bmatrix}
\phi(x,y)-\psi_3(x,y)-2\alpha\cos(2x)\sin(x+y)\\
-\phi(y,x)-\psi_3(x,y)\end{bmatrix}
+\frac{1}{\Lambda}\begin{bmatrix}
\psi_1(x,y)-\psi_2(x,y)\\
\psi_1(x,y)-\psi_2(y,x)\end{bmatrix}.
\]
%%%%%%%%%%%%%%%%%%%%%%%%%%%%%%%%%%%%%%%%%%%%%%%%%%%%%%%%%%%%%%%%%%%%%%%%%%%%%%%
\end{document}
