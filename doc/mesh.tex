\documentclass[a4paper,10pt]{article}
\usepackage[utf8]{inputenc}
\usepackage[margin=2cm]{geometry}
\usepackage{amsmath,graphicx,tikz}
\usepackage[capitalize]{cleveref}
\usepackage{listings}
\lstset{basicstyle={\small\ttfamily}}
%opening
\title{Mesh Data Structures}
\author{Bill McLean}
\date{\today}
%%%%%%%%%%%%%%%%%%%%%%%%%%%%%%%%%%%%%%%%%%%%%%%%%%%%%%%%%%%%%%%%%%%%%%%%%%%%%%%
\begin{document}
\maketitle
%%%%%%%%%%%%%%%%%%%%%%%%%%%%%%%%%%%%%%%%%%%%%%%%%%%%%%%%%%%%%%%%%%%%%%%%%%%%%%%
\section{Meshes for conforming methods}
The \texttt{spatial\_domains} directory contains several geometry description 
files in the gmsh \texttt{.geo} format.  A simple example is \texttt{rect.geo},
which describes a rectangular domain $(0,L_x)\times(0,L_y)$.
\lstinputlisting{../spatial_domains/rect.geo}
The julia commands
\begin{lstlisting}
using FinElt

gmodel = GeometryModel("relative_path_to/rect.geo")
hmax = 0.5
mesh = FEMesh(gmodel, hmax, save_mesh_file=true)
\end{lstlisting}
read the file \texttt{rect.geo} and create a \texttt{GeometryModel} object 
\texttt{gmodel} and a \texttt{FEMesh} object \texttt{mesh}.  The 
\texttt{FEMesh} function also writes the file \texttt{rect.msh} to the
current directory, so you can display the mesh using \texttt{gmsh}; see  
the top three meshes in~\cref{fig: elts}.  (In the gmsh ``Options'' window,
select ``Mesh'' and then, in the ``Visibility'' tab, check the boxes ``2D
element edges'' and one of ``Node labels'', ``1D element labels'' or ``2D
element labels'').

\begin{figure}
\caption{The triangulation with the node and element labels.}\label{fig: elts}
\begin{center}
\includegraphics[scale=0.4]{images/elts0d-crop.pdf}
\end{center}
\vspace{0.1cm}
\begin{center}
\includegraphics[scale=0.4]{images/elts1d-crop.pdf}
\end{center}
\vspace{0.1cm}
\begin{center}
\includegraphics[scale=0.4]{images/elts2d-crop.pdf}
\end{center}
\vspace{0.1cm}
\begin{center}
\includegraphics[scale=0.4]{images/ncelts0d-crop.pdf}
\end{center}
\end{figure}

We find that \texttt{mesh.elt\_nodes\_in} is a Dict whose keys are the physical 
labels defined in \texttt{rect.geo}.  For example, 
\begin{lstlisting}
mesh.elt_tags_in["Left"]
\end{lstlisting}
is the vector \texttt{[24, 25, 26, 27, 28]} of elements along the left 
boundary, and
\begin{lstlisting}
mesh.elt_node_tags_in["Left"]
\end{lstlisting}
is 
\begin{lstlisting}
2x5 Matrix{Int64}:
  4  25  26  27  28
 25  26  27  28   1
\end{lstlisting}
Each column of this matrix gives the node labels of the two end points of 
the corresponding 1D element.  Similarly, if we do
\begin{lstlisting}
elt = mesh.elt_tags_in["Omega"]
node = mesh.elt_node_tags_in["Omega"]
idx = findfirst(elt) do item
    item == 100
end 
\end{lstlisting}
so that \texttt{elt[idx]} equals \texttt{100}, then \texttt{node[:,idx]} is the 
vector \texttt{[3, 17, 62]} of nodes that form the vertices of 
element~\texttt{100} (look in the top right corner).

If we want to solve a boundary-value problem on the rectangular domain, we need 
to specify any essential boundary conditions.   For example, 
\begin{lstlisting}
essential_bc = [("Left", 0.0), ("Right", 1.0)] 
dof = DegreesOfFreedom(mesh, essential_bc)
\end{lstlisting}
creates a \texttt{dof} object that assigns an enumeration of the degrees of
freedom such that the free nodes preceed the fixed nodes.  We find that the
number of free nodes, \texttt{dof.num\_free}, is 57, and the number of fixed
nodes, \texttt{dof.num\_fixed}, is 12.  The labels of the free nodes are given
by the vector \texttt{dof.node\_tag[1:57]}, and of the fixed nodes by
\texttt{dof.node\_tag[58:end]}, the latter being \texttt{[1, 2, 3, 4, 13, 14, 
15, 16, 25, 26, 27, 28]}.

%%%%%%%%%%%%%%%%%%%%%%%%%%%%%%%%%%%%%%%%%%%%%%%%%%%%%%%%%%%%%%%%%%%%%%%%%%%%%%%
\section{Meshes for nonconforming methods}
In our non-conforming method, the degrees of freedom in an element are the 
values of the solution at the midpoints of the edges.  We modify the earlier 
commands as follows.
\begin{lstlisting}
import FinElt.NonConformingPoisson: ELT_DOF
mesh = FEMesh(gmodel, hmax, order=2, save_msh_file=true)
dof = DegreesOfFreedom(mesh, essential_bc, ELT_DOF)
\end{lstlisting}
Here, \texttt{mesh.elt\_type\_in} is 
\begin{lstlisting}
Dict{String, Int32} with 5 entries:
  "Right"  => 8
  "Left"   => 8
  "Omega"  => 9
  "Bottom" => 8
  "Top"    => 8
\end{lstlisting}
and by checking \texttt{mesh.elt\_properties} we see that the
1D and 2D elements are of type \texttt{Line 3} and \texttt{Triangle 6}, 
respectively. The dictionary \texttt{ELT\_DOF} is
\begin{lstlisting}
Dict{Int32, Vector{Int64}} with 2 entries:
  9 => [4, 5, 6]
  8 => [3]
\end{lstlisting}
meaning that in the \texttt{Line 3} elements the degree of freedom is at node~3,
and in the \texttt{Triangle 6} elements the degrees of freedom are at nodes 4, 
5~and 6; see the bottom mesh in~\cref{fig: elts}.

\begin{figure}
\caption{Numbering of the vertices and midpoints in 1D and 2D 
elements.}\label{fig: vertex midpt}
\begin{center}
\begin{tikzpicture}[scale=0.4]
\draw[-] (-13,3) -- (-9,3) -- (-5,3);
\foreach \x in {-13, -9, -5}
    \draw[fill] (\x,3) circle(0.1);
\node[below] at (-13,3) {1};
\node[below] at (-9,3) {3};
\node[below] at (-5,3) {2};
\draw[-] (0,0) -- (6,4) -- (-2,8) -- (0,0);
\draw[fill] (0,0) circle (0.1);
\node[below] at (0,0) {1};
\draw[fill] (6,4) circle (0.1);
\node[right] at (6,4) {2};
\draw[fill] (-2,8) circle (0.1);
\node[above] at (-2,8) {3};
\draw[fill] (3,2) circle (0.1);
\node[below right] at (3,2) {4};
\draw[fill] (2,6) circle (0.1);
\node[above right] at (2,6) {5};
\draw[fill] (-1,4) circle (0.1);
\node[above right] at (-1,4) {6};
\end{tikzpicture}
\end{center}
\end{figure}

The numbering of the 1D and 2D elements has not changed, but the nodes labels 
are now as shown in \cref{fig: ncelts}.  Repeating the commands
\begin{lstlisting}
elt = mesh.elt_tags_in["Omega"]
node = mesh.elt_node_tags_in["Omega"]
idx = findfirst(elt) do item
    item == 100
end 
\end{lstlisting}
we now find that \texttt{node[:,idx]} is \texttt{[3, 31, 90, 39, 208, 209]}, 
the new vector of node labels in element~\texttt{100}.  The number of free 
nodes, \texttt{dof.num\_free}, is 166, and the number of fixed nodes,
\texttt{dof.num\_fixed}, is 10.  The labels of the fixed nodes are given by
\texttt{dof.node\_tag[167:end]}, which is the vector \texttt{[26, 27, 28, 29, 
30, 52, 53, 54, 55, 56]} consisting of the midpoint nodes in the 1D elements 
along the left and right sides.


%%%%%%%%%%%%%%%%%%%%%%%%%%%%%%%%%%%%%%%%%%%%%%%%%%%%%%%%%%%%%%%%%%%%%%%%%%%%%%%
\end{document}
